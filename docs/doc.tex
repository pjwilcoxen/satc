\documentclass[12pt]{article}

\usepackage{hyperref}
\hypersetup{
    colorlinks=true,
    linktoc=all,
    linkcolor=black,
    urlcolor=cyan,
}

\usepackage{listings}
\usepackage{color}
\definecolor{codegreen}{rgb}{0,0.6,0}
\definecolor{codegray}{rgb}{0.5,0.5,0.5}
\definecolor{codepurple}{rgb}{0.58,0,0.82}
\definecolor{backcolour}{rgb}{0.95,0.95,0.92}
\lstdefinestyle{mystyle}{
    backgroundcolor=\color{backcolour},   
    commentstyle=\color{codegreen},
    keywordstyle=\color{magenta},
    numberstyle=\tiny\color{codegray},
    stringstyle=\color{codepurple},
    basicstyle=\footnotesize,
    breakatwhitespace=false,         
    breaklines=true,                 
    captionpos=b,                    
    keepspaces=true,                 
    numbers=left,                    
    numbersep=5pt,                  
    showspaces=false,                
    showstringspaces=false,
    showtabs=false,                  
    tabsize=2
}
\lstset{style=mystyle}

\title {SaTC Reference Manual}
\author {A}

\begin{document}

\maketitle
\newpage
\tableofcontents

%% Overview
\newpage

\section{Overview}

%% Installation
\section {Installation} 
\label{sec:installation}

SaTC project depends on two external libraries: 

\begin{itemize}
    \item{\textbf{commons-csv-1.4.jar}} \\
        Commons CSV reads and writes files in variations of the Comma Separated Value (CSV) format. 
        More details can be found on the project official website: 
        \href{https://commons.apache.org/proper/commons-csv/}{Commons CSV Home}
    \item{\textbf{mason.19.jar}} \\
        SaTC project uses the MASON discrete event simulation framework to manage steps happening 
        in the dynamic distributed market simulator. 
        More details can be found on the project official website: 
        \href{https://cs.gmu.edu/~eclab/projects/mason/}{Mason}
\end{itemize}

\noindent
To build the SaTC project: 

\begin{itemize}
    \item{If you have the build tool `make', }
    you can use the command line to enter the directory `src' and then execute the command `make.' 
    \item{If you do not have any tool that is able to process `makefile', }
    you need execute the following commands manually: 

    \begin{enumerate}
        \item{Launch the command line and change the working directory to the directory `src'.}
        \item{Execute the following command to compile the java code: } 
            \begin{center}
                \lstinline[language=bash]!javac -cp '../lib/*;.' $*.java!
            \end{center}
        \item{Generate the java archive file with the following command: }
            \begin{center}
                \lstinline[language=bash]!jar cvfm market.jar manifest.txt *.class!
            \end{center}
        \item{After running the commands above, you can obtain the archive file `market.jar' in the `src' directory.}
    \end{enumerate}

\end{itemize}

%% Running
\section{Running} 
\label{sec:running}

\subsection{Run the Simulation}
\label{subsec:running_runsimulation}

You should build the SaTC project firstly before running the simulator. 
Build reference can be found in the section `\hyperref[sec:installation]{Installation}'. 

\bigskip
\noindent
If you have built the SaTC project successfully, 
you should be able to find a Java archive file `market.jar' in the directory `src'. 
Then you can follow the steps below to run the simulation:

\begin{enumerate}
    \item{Launch the command line and change the working directory to the directory `run'.}
    \item{Execute the following command: }
        \begin{center}
            \lstinline[language=bash]!java -jar ../src/market.jar <run_configuration_file>!
        \end{center}
        More details about the run configuration files can be found in the section: 
        \hyperref[subsec:running_runconfig]{Run Configuration Files}
\end{enumerate}

%% Running - Run configuration files
\subsection{Run Configuration Files}
\label{subsec:running_runconfig}

Run configuration files provides environment configurations for the simulator, 
describing environmental information that will be used in the simulation. 

\bigskip
\noindent
\begin{tabular} {| l | l | l |}
    \hline
    \textbf{option} & \textbf{meaning} & \textbf{} \\ \hline
    netmap & Path of the grid agent configuration file. & Required \\ \hline
    virtualmap & Path of the virtual agent configuration file. & Optional \\ \hline
    draws & Path of the draw file. & Required \\ \hline
    history & Path of the historical output file. & Optional \\ \hline
    bids & Path of the bid file. & Optional \\ \hline
    transcost & Transmission cost between nodes. & Optional \\ \hline
    transcap & Maximum transmission between nodes. & Optional \\ \hline
    seed & Seed for random number generator. & Required \\ \hline
    debug & Debugging flag (0, 1, default: 0). & Required \\ \hline
\end{tabular}

\bigskip
\noindent
Note that if you set the `transcost' and `transcap' in the run configuration file, 
then their values will be appied to all the grid agents in the market. 
Cost and capacity settings for individual agent in the file `netmap.csv' will be ignored. 

\bigskip
\noindent
\paragraph{Example}
The following example is from the configuration file `c2k1800.txt'.  
This file is used for simulation without any adversary. 
In this case, you do not need specify `virtualmap', `history', and `bids' files 
since these files are only used by virtual agents. 

\bigskip
\noindent
\begin{lstlisting}
netmap    : netmap.csv
draws     : testdraw.csv
transcost : 2
transcap  : 1800 
seed      : 123456
debug     : 1
\end{lstlisting}

\bigskip
\noindent
This configuration file indicates that `netmap.csv' and `testdraw.csv' in the current directory 
will be used as grid agent configuration and draw file, respectively. 
Transmission cost between two nodes will be 2 and maximum transmission between two nodes will be 1800. 
`123456' will work as seed for random number generator. 
Debugging messages will also be enabled in this simulation.  

\bigskip
\noindent
\paragraph{Example}
The following example is from the configuration file `c2k1800d.txt'. 
This file is used for simulation with adversary `Adv\_Darth'. 
You have to specify `virtualmap', `history', and `bids' files 
for the adversary `Darth'. 

\bigskip
\noindent
\begin{lstlisting}
netmap      : netmapd.csv
virtualmap  : virtualmapd.csv
draws       : testdraw.csv
history     : c2k1800_out.csv
bids        : c2k1800_net.csv
transcost   : 2
transcap    : 1800 
seed        : 123456
debug       : 1
\end{lstlisting}

\bigskip
\noindent
Note that compared with the previous example, this configuration file 
contains three more options: 
\begin{itemize}
    \item{virtualmap : virtualmapd.csv}
    \item{history    : c2k1800\_out.csv}    
    \item{bids       : c2k1800\_net.csv}
\end{itemize}
These options indicate the path to the `virtualmap' file, `history' file, 
and `bids' file, respectively.

%% Input
\section{Input}
\label{sec:input}

%% Input - netmap.csv
\subsection{netmap.csv} 
\label{subsec:input_netmap}

`Netmap' file provides grid agent configuration for the simulator, 
describing details of each grid agent in the electricity market. 

\bigskip
\noindent
\begin{tabular} {| l | l |}
    \hline
    \textbf{column} & \textbf{meaning} \\ \hline
    id & Unique identifier of the grid agent. \\ \hline
    type & 
    \vtop{\hbox{\strut Grid agent's role in the electricity market. } 
    \hbox{\strut (1: Level 1 market, top level; 2: Level 2 market; 3: Traders.) }} \\ \hline
    sd\_type & Grid agent's type as supplier (S) or consumer (D). \\ \hline
    up\_id & ID of the upstreaming node that the current node connects to.\\ \hline
    channel & Communication channel used by the current node. \\ \hline
    cost & Transmission cost. \\ \hline
    cap & Transmission capacity. \\ \hline
    security & Security level (0-100) of the current grid agent. \\ \hline
\end{tabular}

\bigskip
\noindent
\paragraph{Example}
The following example is the configuration of node 1: \\
\begin{lstlisting}
1,3,D,201,type1,1,2000,100
\end{lstlisting}

\bigskip
\noindent
Explanation: 

\bigskip
\noindent
\begin{tabular} {| l | l | l |}
    \hline
    \textbf{column} & \textbf{value} & \textbf{meaing} \\ \hline
    id & 1 & This grid agent's id is 1. \\ \hline
    type & 3 & This grid agent is a trader working on the level 3. \\ \hline
    sd\_type & D & This grid agent is a consumer. \\ \hline
    up\_id & 201 & This grid agent connects to upstreaming node 201. \\ \hline
    channel & type1 & This grid agent is using type1 channel. \\ \hline
    cost & 1 & The transmission cost on this wire is 1. \\ \hline
    cap & 2000 & The transmission capacity of this wire is 2000. \\ \hline
    security & 100 & This grid agent's security level is 100. \\ \hline
\end{tabular}

%% Input - virtualmap.csv
\subsection{virtualmap.csv}
\label{subsec:input_virtualmap}

`Virtualmap' file provides virtual agent configuration for the simulator, 
describing details of each virtual agent in the electricity market. 

\bigskip
\noindent
\begin{tabular} {| l | l |}
    \hline
    \textbf{column} & \textbf{meaning} \\ \hline
    id & Unique identifier of virtual agents. \\ \hline
    type & 
    \vtop{\hbox{\strut Virtual agent type, including Adv\_Adam, Adv\_Beth, } 
    \hbox{\strut Adv\_Darth, Adv\_Elvira, Adv\_Faust. }} \\ \hline
    configuration & Configuration parameters for virtual agent's behavior. \\ \hline
    channel & Communication channels the virtual agent has access to. \\ \hline
    agent & ID of grid agents that virtual agent has access to. \\ \hline
    intel\_level & 
    \vtop{\hbox{\strut How much information the virtual agent has. }
    \hbox{\strut `full' means the virtual agent has intel from all the grid agents. }
    \hbox{\strut `partial' means the virtual agent has intel from the grid agents} 
    \hbox{\strut in its intel set. }} \\ \hline
    intel & Set of information that virtual agent has \\ \hline
    security & KC \\ \hline
\end{tabular}

\bigskip
\noindent
The table below introduces configuration parameters used by virtual agents. 

\bigskip
\noindent
\begin{tabular} {| l | l |}
    \hline
    \textbf{configuration} & \textbf{meaning} \\ \hline

    capability & 
    \vtop{\hbox{\strut Integer. Used by adversaries. }
    \hbox{\strut It represents an adversary's capability to compromise grid agents. }
    \hbox{\strut An adversary can only compromise those grid agents with }
    \hbox{\strut a lower security than the adversary's capability. }} \\ \hline

    target & 
    \vtop{\hbox{\strut Integer, market agent's ID. }
    \hbox{\strut Used by adversaries: Adv\_Darth, Adv\_Elvira, and Adv\_Faust. }
    \hbox{\strut It indicates the market that will be attacked. }} \\ \hline

    trader &
    \vtop{\hbox{\strut Integer, trader agent's ID. }
    \hbox{\strut Used by the adversary Adv\_Darth. }
    \hbox{\strut It indicates the trader that will be compromised for attack. }} \\ \hline 

    shift &
    \vtop{\hbox{\strut Integer. }
    \hbox{\strut Used by adversaries: Adv\_Darth, Adv\_Elvira. }
    \hbox{\strut PW }} \\ \hline

    reduction &
    \vtop{\hbox{\strut Integer. Range 1-99. }
    \hbox{\strut Used by the adversary Adv\_Faust. }
    \hbox{\strut It indicates the percentage amount by which the bids price }
    \hbox{\strut will be decreased when the adversary forges fake messages.}} \\ \hline
\end{tabular}

\bigskip
\noindent
\paragraph{Example}
The following example is the configurations of virtual agent 
`Adv\_Beth' from file `virtualmapb.csv'. 
\bigskip
\noindent
\begin{lstlisting}
1001,Adv_Beth,capability:0,"type1,type2,type3","1,2,3,4,5,201",partial,"1,2,3,4,5,201",100
\end{lstlisting}

\bigskip
\noindent
Explanation: 

\bigskip
\noindent
\begin{tabular} {| l | l | l |}
    \hline
    \textbf{column} & \textbf{value} & \textbf{meaning} \\ \hline
    id & 1001 & This virtual agent's id is 1001. \\ \hline
    type & Adv\_Beth & This virtual agent is adversary `Adv\_Beth'. \\ \hline
    configuration & capability:0 &
    \vtop{\hbox{\strut This virtual agent's capability is 0, which means}
    \hbox{\strut it is not able to compromise any grid agent.}} \\ \hline
    channel & type1,type2,type3 & 
    \vtop{\hbox{\strut This virtual agent has access to the type1, }
    \hbox{\strut type2, and type3 communication channel.}} \\ \hline
    agent & 1,2,3,4,5,201 & 
    \vtop{\hbox{\strut This virtual agent has access to the }
    \hbox{\strut grid agents 1, 2, 3, 4, 5, and 201.}} \\ \hline
    intel\_level & partial & 
    \vtop{\hbox{\strut This virtual agent is able to obtain intel }
    \hbox{\strut from the grid agents inside its intel set.}} \\ \hline
    intel & 1,2,3,4,5,201 & 
    \vtop{\hbox{\strut This virtual agent has information from the }
    \hbox{\strut grid agents 1, 2, 3, 4, 5, and 201.}} \\ \hline
    security & 100 & KC \\ \hline
\end{tabular}

\bigskip
\noindent
\paragraph{Example}
The following example is the configurations of virtual agent 
`Adv\_Darth' from file `virtualmapd.csv'. 
\bigskip
\noindent
\begin{lstlisting}
1001,Adv_Darth,"capability:100,target:202,trader:191,shift:100","type1,type2","191",full,"191,202,203",100,none
\end{lstlisting}

\bigskip
\noindent
Explanation: 

\bigskip
\noindent
\begin{tabular} {| l | l | l |}
    \hline
    \textbf{column} & \textbf{value} & \textbf{meaning} \\ \hline
    id & 1001 & This virtual agent's id is 1001. \\ \hline
    type & Adv\_Darth & This virtual agent is adversary `Adv\_Darth'. \\ \hline
    configuration 
    & 
    \vtop{\hbox{\strut capability:100,}
    \hbox{\strut target:202,}
    \hbox{\strut trader:191,}
    \hbox{\strut shift:100}}
    &
    \vtop{\hbox{\strut This virtual agent's capability is 100. }
    \hbox{\strut This virtual agent will attack market agent 202. }
    \hbox{\strut This virtual agent will try to compromise trader 191. }
    \hbox{\strut The demand curve will be shifted by 100. }} \\
    \hline
    channel & type1,type2 & 
    \vtop{\hbox{\strut This virtual agent has access to the  }
    \hbox{\strut type1 and type2 communication channel.}} \\ \hline
    agent & 191 & This virtual agent has access to the grid agent 191. \\ \hline
    intel\_level & full & 
    \vtop{\hbox{\strut This virtual agent is able to obtain intel }
    \hbox{\strut from all the grid agents.}} \\ \hline
    intel & 191,202,203 & 
    \vtop{\hbox{\strut This virtual agent has information from the }
    \hbox{\strut grid agents 191, 202, and 203.}} \\ \hline
    security & 100 & KC \\ \hline
\end{tabular}

\bigskip
\noindent
\paragraph{Example}
The following example is the configurations of virtual agent 
`Adv\_Elvira' from file `virtualmape1.csv'. 
\bigskip
\noindent
\begin{lstlisting}
1001,Adv_Elvira,"capability:100,target:202,shift:100","type1,type2",191,full,"191,202",100,none
\end{lstlisting}

\bigskip
\noindent
Explanation: 

\bigskip
\noindent
\begin{tabular} {| l | l | l |}
    \hline
    \textbf{column} & \textbf{value} & \textbf{meaning} \\ \hline
    id & 1001 & This virtual agent's id is 1001. \\ \hline
    type & Adv\_Elvira & This virtual agent is adversary `Adv\_Elvira'. \\ \hline
    configuration 
    & 
    \vtop{\hbox{\strut capability:100,}
    \hbox{\strut target:202,}
    \hbox{\strut shift:100}}
    &
    \vtop{\hbox{\strut This virtual agent's capability is 100. }
    \hbox{\strut This virtual agent will attack market agent 202. }
    \hbox{\strut The demand curve will be shifted by 100. }} \\
    \hline
    channel & type1,type2 & 
    \vtop{\hbox{\strut This virtual agent has access to the  }
    \hbox{\strut type1 and type2 communication channel.}} \\ \hline
    agent & 191 & This virtual agent has access to the grid agent 191. \\ \hline
    intel\_level & full & 
    \vtop{\hbox{\strut This virtual agent is able to obtain intel }
    \hbox{\strut from all the grid agents.}} \\ \hline
    intel & 191,202 & 
    \vtop{\hbox{\strut This virtual agent has information from the }
    \hbox{\strut grid agents 191 and 202.}} \\ \hline
    security & 100 & KC \\ \hline
\end{tabular}

\bigskip
\noindent
\paragraph{Example}
The following example is the configurations of virtual agent 
`Adv\_Faust' from file `virtualmapf.csv'. 
\bigskip
\noindent
\begin{lstlisting}
1001,Adv_Faust,"capability:100,target:202,reduction:30","type1,type2","202",full,"202",100,none
\end{lstlisting}

\bigskip
\noindent
Explanation: 

\bigskip
\noindent
\begin{tabular} {| l | l | l |}
    \hline
    \textbf{column} & \textbf{value} & \textbf{meaning} \\ \hline
    id & 1001 & This virtual agent's id is 1001. \\ \hline
    type & Adv\_Faust & This virtual agent is adversary `Adv\_Faust'. \\ \hline
    configuration 
    & 
    \vtop{\hbox{\strut capability:100,}
    \hbox{\strut target:202,}
    \hbox{\strut reduction:30}}
    &
    \vtop{\hbox{\strut This virtual agent's capability is 100. }
    \hbox{\strut This virtual agent will attack market agent 202. }
    \hbox{\strut Bid prices will be decreased by 30\%. }} \\
    \hline
    channel & type1,type2 & 
    \vtop{\hbox{\strut This virtual agent has access to the  }
    \hbox{\strut type1 and type2 communication channel.}} \\ \hline
    agent & 202 & This virtual agent has access to the grid agent 202. \\ \hline
    intel\_level & full & 
    \vtop{\hbox{\strut This virtual agent is able to obtain intel }
    \hbox{\strut from all the grid agents.}} \\ \hline
    intel & 202 & This virtual agent has information from the agent 202. \\ \hline
    security & 100 & KC \\ \hline
\end{tabular}


%% Input - testdraw.csv
\subsection{testdraw.csv}
\label{subsec:input_testdraw}

\begin{tabular} {| l | l |}
    \hline
    \textbf{column} & \textbf{meaning} \\ \hline
    n & PW \\ \hline
    type & PW \\ \hline
    load & PW \\ \hline
    elast & PW \\ \hline
\end{tabular}

%% Output
\section{Output}
\label{sec:output}

%% Output - Log files
\subsection{Log Files}
\label{subsec:output_log}

A log file records all the messages reported by the simulator during runtime. 
It contains: 
\begin{itemize}
    \item{Configuration of this scenario.}
    \item{Steps that have been reached in each population.}
    \item{Results after each population.}
    \item{Bids dropped in each population.}
    \item{Some other debugging messages.}
\end{itemize}

%% Output - Net demand files / debugging files
\subsection{Net Demand Files / Debugging Files}
\label{subsec:output_netdemand}

\bigskip
\noindent
\begin{tabular} {| l | l |}
    \hline
    \textbf{column} & \textbf{meaning} \\ \hline
    pop & PW \\ \hline
    id & Unique identifier of the grid agent. \\ \hline
    tag & PW \\ \hline
    dos & PW \\ \hline
    steps & Demand steps the grid agent has in one dos from each population. \\ \hline
    p & Price of the current demand step. \\ \hline
    q\_min & Minimum quantity of the current demand step. \\ \hline
    q\_max & Maximum quantity of the current demand step. \\ \hline
\end{tabular}

%% Output - Output files
\subsection{Output Files}
\label{subsec:output_output}

`Output' file records grid agent's price, actual quantity, upstreaming transmission constraint, 
and some other runtime information after each simulation. 

\bigskip
\noindent
\begin{tabular} {| l | l |}
    \hline
    \textbf{column} & \textbf{meaning} \\ \hline
    pop & PW \\ \hline
    dos & PW \\ \hline
    id & Unique identifier of the grid agent. \\ \hline
    rblock & PW \\ \hline
    blocked & PW \\ \hline
    p & Grid agent's price record after one simulation. \\ \hline
    q & Grid agent's actual quantity record after one simulation. \\ \hline
    upcon & 
    \vtop{\hbox{\strut If the grid agent's upstream transmission is binding or not.} 
    \hbox{\strut (N: No constraint; S: At maximum supply; D: At maximum demand.) }} \\ \hline
\end{tabular}

\bigskip
\noindent
\paragraph{Example}
The following example is an output record from the output file `c2k1800\_out.csv'.
\bigskip
\noindent
\begin{lstlisting}
1,0,1,41.3,0,70,40,N
\end{lstlisting}

\bigskip
\noindent
Explanation: 

\bigskip
\noindent
\begin{tabular} {| l | l | l |}
    \hline
    \textbf{column} & \textbf{value} & \textbf{meaning} \\ \hline
    pop & 1 & This record comes from population 1. \\ \hline
    dos & 0 & This record got after dos 0. \\ \hline
    id & 1 & This record belongs to the grid agent 1. \\ \hline
    rblock & 41.3 & PW \\ \hline
    blocked & 0 & PW \\ \hline
    p & 70 & The price of grid agent 1 is 70. \\ \hline
    q & 40 & The actual quantity of grid agent 1 is 40. \\ \hline
    upcon & N & No constraint exists in the upstream transmission. \\ \hline
\end{tabular}

%% Modeling Framework
\section{Modeling Framework}
\label{sec:modeling}

%% Modeling Framework - Simulation Environment
\subsection{Simulation Environment} \mbox{}

Simulator firstly loads run configurations into the simulation environment. 
\hyperref[subsec:objects_env]{`Env'} object will check settings from the configuration file, 
then initiate agents in the market model using corresponding data files. 

Environment is also responsible for storing critical variables and information for the simulation. 
Some of them are defined as `static' in order that these data can be shared with other objects. 

Stage change is also controlled by the environment. 
It starts up the simulation from the stage SERVICE\_SEND, then to the stage TRADER\_SEND, PRE\_AGGREGATE, 
AGGREGATE, PRE\_REPORT, REPORT, PRE\_CALC\_LOADS and the simulation eventually terminates at the stage CALC\_LOADS. 

Different kinds of agents work in different stages: 
\begin{itemize} 
    \item{\textbf{Trader agents}} send demand messages to the upstream market during the stage TRADER\_SEND, 
    and receive price messages and determine the loads during the stage CALC\_LOADS. 
    \item{\textbf{Market agents}} aggregate demand messages from the downstream agents and 
    send it to the upstream market at the stage AGGREGATE. 
    Then they will receive price messages from the upstream market and report the price to downstream agents 
    during the stage REPORT. 
    And they also calculate the actual loads during the stage CALC\_LOADS.
    \item{\textbf{Adversaries}} can hook into the `PRE' stages to intercept the channels and send fake messages. 
    These `PRE' stages happen before the corresponding stages and are only used to simulate attack behaviors.
\end{itemize}

More details can be found in the section `\hyperref[subsec:objects_env]{Env}'. 

// Env

%% Modeling Framework - Traders and Markets
\subsection{Traders and Markets} \mbox{}

Traders and markets are grid agent nodes connected by both physical connections and virtual connections. 
Trader agent can work as a demander to consume electricity or as a supplier to provide electricity. 
Traders interact with their upstream market using communication channels. 

Market agent is responsible for demand aggregation and electricity rates determination. 
There are several tiers of market agents in the model: 
\begin{itemize}
    \item{\textbf{Root market agent}} calculates the electricity price based on the demand messages 
    from the downstream child agents. 
    \item{\textbf{Intermediate market agent}} utilizes the price message from its upstream market and 
    other transmission parameters, such as transmission cost and capacity limit, 
    to determine the actual price, and sends the actual price to downstream child agents.  
\end{itemize}

More details can be found in the section `\hyperref[subsec:objects_trader]{Trader}', 
`\hyperref[subsec:objects_market]{Market}', and `\hyperref[subsec:objects_demand]{Demand}'. 

// Trader, market, demand, net demands 

%% Modeling Framework - Price Determination
\subsection{Price Determination} \mbox{}

Transmission costs, congestion, aggregation, population

%% Modeling Framework - Communications
\subsection{Communications} \mbox{}

Each grid agent in the market model does not only has physical connection to 
its upstream market agent but also has a virtual connection, which is called communication channel, 
with the upstream object. 
Note that this is not applied to the market agent on the top level since it does not have an upstream market agent. 

The physical connection is used to simulate transmission lines and 
the communication channel is for transmitting messages between grid agents and the upstream markets. 
Three types of communication channels are being used in the simulation: type1, type2, and type3 channel. 
Virtual agents are able to connect with the market and send messages through communication channels as well.

There are three kinds of messages transmitted through channels: empty, demand, and price message. 
Demand messages are sent by a grid agent to its upstream market. 
Price messages come from upstream market to downstream agents. 

More details can be found in the section `\hyperref[subsec:objects_channel]{Channel}' and
`\hyperref[subsec:objects_msg]{Msg}'. 

// Channel, msg 

%% Modeling Framework - DoS
\subsection{DoS} \mbox{}

DoS is the abbreviation standing for the term, `Denial of Service'. 
We simulate this process by randomly dropping bids during message transmission through channels. 
The percentage of the dropped bids is given by the configuration file. 
Note that this attack is more severe in the real world with a higher bids dropped rate.

%% Modeling Framework - Adversaries
\subsection{Adversaries} \mbox{}

A virtual agent is a special market node that does not have physical connections to other grid agents 
but is able to build virtual connections with grid agents through communication channels. 

Adversary is a kind of virtual agent that can access senstive information and 
perform attacks based on these information. 
They can access historical bids, demands data, and other intel from some grid agents, 
and even compromise the grid agent. 
They can also interfere the system by intercepting the channels and steal runtime information, 
such as message being transmitted in the channel. 
Then they utilize these information to forge messages and spread the fake messages 
in the market to destruct the electricity and price control. 

More details can be found in the section `\hyperref[subsec:objects_adversary]{Adversary}', 
`\hyperref[subsec:objects_history]{History}', and `\hyperref[subsec:objects_intel]{Intel}'. 

// History, intel

%% Objects
\section{Objects Reference}
\label{sec:objects}

%% Objects - Adversary
\subsection{Adversary} \mbox{}
\label{subsec:objects_adversary}

Adversary is an abstract class for virtual agents who try to disrupt the
power grid. It has several subclasses that can perform different
kinds of attacks.

\paragraph{Adv\_Adam} \mbox{}

An \textbf{Adv\_Adam} object can send false bids to the other nodes inside the grid. 
It looks up the \textbf{Intel} hash map and picks all the target nodes that be 
connected to. Then constructs the false bid and sends the bid to the targets.

\paragraph{Adv\_Beth} \mbox{}

\textbf{Adv\_Beth} class is the promoted version for \textbf{Adv\_Adam} class. It has all 
the functionalities provided by \textbf{Adv\_Adam} as well as to forge credentials 
and to decipt the recipient with that.

\paragraph{Adv\_Darth} \mbox{}

An \textbf{Adv\_Darth} object is able to attack the market with constrained transmission 
to their upstream nodes. It is supposed to have compromised one trader from the 
target market. It first intercepts all messages sent by the compromised trader. 
Then extracts the trader's historical demands. Shifts the demand curve by the attacker 
customized distance. Finally injects the fake demands back into the channel and these 
demands will be sent to the target market. 

\paragraph{Adv\_Elvira} \mbox{}

\textbf{Adv\_Elvira} performs similar behavior to \textbf{Adv\_Darth}. We suppose that \textbf{Adv\_Elvira}
has compromised several traders instead of only one. Then it performs the attack from 
all of these compromised traders, and shifts all their demand curves by a smaller value, 
with a total shift distance that equals to the attacker customized distance. Compared 
with the \textbf{Darth}, \textbf{Elvira} performs the attack more softly but with more traders to achieve 
the same result, which means it has less possbility to be discovered by the anomalous data 
detection.

\paragraph{Adv\_Faust} \mbox{}

An \textbf{Adv\_Faust} object listens the target market channel and intercepts messages 
sent by the target. Then injects the DEMAND message back to the channel without any
modification, but tampers the price info in the PRICE message and injects it back 
to the channel before traders calculate loads. 

%% Objects - Agent
\subsection{Agent} \mbox{}
\label{subsec:objects_agent}

Agent is an abstract class that provides basic features for entities that 
communicate.  It has two subclasses: \textbf{Grid}, for agents actually connected
to the power grid, and \textbf{Virtual}, for agents that only have communication 
links.

%% Objects - Channel
\subsection{Channel} \mbox{}
\label{subsec:objects_channel}

A \textbf{Channel} object is communications channel.  An arbitrary number are 
allowed and they can have different properties.  Each agent has a specific 
Channel that it uses to communicate with its upstream parent node.  A Channel 
is used to by one agent to send Msg objects to another.

Each Channel has one main method, \textbf{send()}.  It looks up the sender and 
recipient from the corresponding fields of the Msg object it is given and checks 
whether random denial of service filtering applies to the sender.  If so, 
the message is dropped. Otherwise, as long as the message is not diverted (discussed
below) it is passed to the recipient via the recipient's \textbf{deliver()} method.

Three hooks are available to support man in the middle attacks and other 
interventions.  Message diversions can be set up via each channel's 
\textbf{divert\_to()} and \textbf{divert\_from()} methods. The first diverts all 
messages sent \textit{to} a given node and the second diverts all messages sent
*by* a given node. The \textit{from} diversion is processed first and when both 
apply to a given message it takes precedence.  Messages can be reinserted 
downstream from the diversions via the channel's \textbf{inject()} method.  Future 
features to be implemented:

\begin{itemize}
  \item{Random DOS loss rates that can vary by channel}
  \item{VPN channels that prohibit diversions}
  \item{Authentication of senders}
  \item{Authentication of recipients -- blocking \textbf{divert\_to()}}
\end{itemize}

%% Objects - Demand
\subsection{Demand} \mbox{}
\label{subsec:objects_demand}

A \textbf{Demand} object holds a net demand curve expressed as a list of steps.
Positive quantities indicate demand and negative quantites indicate supply.
\textbf{Trader} nodes send \textbf{Demand} objects to \textbf{Market} nodes.  Lower tier 
\textbf{Market} nodes send aggregated \textbf{Demand} objects to higher-tier \textbf{Market} 
nodes.  In all cases the curves are sent via \textbf{Msg} objects.

%% Objects - Env
\subsection{Env} \mbox{}
\label{subsec:objects_env}

The \textbf{Env} object represents the environment under which the simulation
is running.  It includes various global variables and is also responsible
for loading data, configuring the network of \textbf{Agent} nodes, 
the \textbf{Channel} objects they use to communicate, and then starting the
simulation.

%% Objects - Grid
\subsection{Grid} \mbox{}
\label{subsec:objects_grid}

Abstract class for grid-connected agents (that is, agents through
which power can flow).  Has two subclasses: \textbf{Trader} and \textbf{Market}.

%% Object - History
\subsection{History} \mbox{}
\label{subsec:objects_history}

A \textbf{History} object stores historic price and quantity data for a specific 
agent. Provides several methods used to store and retrieve demand, 
price, quantity, and constraint information. The simulator can load historic 
data from preset history file before it runs. Adversaries can extract 
relevant information from history object to build fake demand curves 
and then perform attacks.

%% Object - Intel
\subsection{Intel} \mbox{}
\label{subsec:objects_intel}

Stores a virtual agent's information about another agent within the grid. 
Contains functionality to store an agent's 
grid-level (transmission cost, price, parent, children, tier), 
historic (price, quantity, bid), and some status (compromised, interceptTo, 
interceptFrom, forge, send ,learned) information. Can also 
retrieve max/min/avg information regarding p and q.

%% Object - Market
\subsection{Market} \mbox{}
\label{subsec:objects_market}

Represents a market.  Aggregates demands by child nodes, which can be 
\textbf{Trader} or other \textbf{Market} nodes. If the node has a parent, it adjusts 
the aggregate demand for transmission costs and capacity to the parent
and passes the adjusted demand up.  If the node does not have a parent, 
it determines the equilibrium price.  Market nodes receive prices from 
upstream, adjust them for transmission parameters, and then pass them 
down to child nodes.

%% Object - Msg
\subsection{Msg} \mbox{}
\label{subsec:objects_msg}

A single message from one agent to another.  At the moment, two types of 
messages can be sent: one with a \textbf{Demand} object and one with a 
price.  All \textbf{Msg} objects are passed via \textbf{Channel} objects. Future 
features to be implemented:

\begin{itemize}
  \item{Encryption that prevents reading by diverters}
  \item{Signing that prevents spoofing the sender}
\end{itemize}

%% Object - Trader
\subsection{Trader} \mbox{}
\label{subsec:objects_trader}

Abstract class for end agents. \textbf{Trader} nodes have upstream parents
that are \textbf{Market} nodes.  They submit \textbf{Demand} objects 
to their parent \textbf{Market} nodes and then receive prices back. 
Final demand or supply results from the prices received. 

Provides method \textbf{getOneDemand()} to retrieve demand, static method 
\textbf{readDraws()} to load draws from history files, and abstract method 
\textbf{drawLoad()} to build the agent's demand curve.

Has subclass: \textbf{TraderMonte}.

\paragraph{TraderMonte} \mbox{}

A \textbf{TraderMonte} object represents the trader under Monte Carlo mode. 
Has two types: end users and suppliers. Implements method \textbf{drawLoad()}, 
and provides method \textbf{readDraws()}.

%% Objects - Util
\subsection{Util} \mbox{}
\label{subsec:objects_util}

A utility class that includes a few general purpose methods for opening
files with built-in exception handling.

%% Objects - Virtual
\subsection{Virtual} \mbox{}
\label{subsec:obejcts_virtual}

Abstract class for agents connected to the communications network but not
connected directly to the power grid.  

\end{document}